\documentclass{report}

\usepackage[ngerman]{babel}
\usepackage[utf8]{inputenc}
\usepackage[T1]{fontenc}
\usepackage{hyperref}
\usepackage{csquotes}
\usepackage[a4paper]{geometry}

\usepackage[
    backend=biber,
    style=apa,
    sortlocale=de_DE,
    natbib=true,
    url=false,
    doi=false,
    sortcites=true,
    sorting=nyt,
    isbn=false,
    hyperref=true,
    backref=false,
    giveninits=false,
    eprint=false]{biblatex}
\addbibresource{../references/bibliography.bib}

\parskip=1em
\parindent=0em
\title{Ethik im Umgang mit Daten}
\author{Levent Bal}
\date{\today}


\begin{document}

\maketitle

\abstract{
In dieser Arbeit werden folgene Fragestellungen besprochen.

\item \textbf{Wie kann Künstliche Intelligenz (KI) in der Bildung eingesetzt werden?}
\begin{itemize}
    \item Diese Frage ist zentral, da KI das Potenzial hat den Bildungssektor zu verändern und zu verbessern. Es ist wichtig zu verstehen wie KI zur Unterstützung von Lehrern und Schülern genutzt werden kann.
\end{itemize}

\item \textbf{Sollte unser Schulsystem angepasst werden?}
\begin{itemize}
    \item Diese Frage zielt darauf ab, zu diskutieren, ob und wie das bestehende Schulsystem verändert werden sollte, um die Integration von KI zu ermöglichen und optimal zu nutzen. 
\end{itemize}

\item \textbf{Weitere Fragen}
\begin{itemize}
    \item Weitere Fragen, wie die Funktionsweise von KI oder die ethischen Aspekte der Nutzung von KI-Informationen, werden ebenfalls in dieser Arbeit behandelt.
\end{itemize}


\tableofcontents

\chapter{Einleitung}

\section{Was ist KI?}
Bei der Entwickluung von Systemen mit Künstlicher Intelligenz streben Forscher danch das menschliche Handeln und denken mit hilfe von einer Maschine nachzubauen. Eine offizielle Definition vom Begriff Ki gibt es aber nicht. Daher wird Ki meistens durch Eigenschaften definiert wie z.B Spracherkennung, Bildverarbeitung und ML.\citep{Produktion-Technik}

\section{Wie funktioniert KI?}

}
\chapter{Ki und Ethik}

\chapter{Ki in der Bildung}





\section{Ki in der Bildung}
Die KI bietet zahlreiche Möglichkeiten, Lernprozesse und Lehrkräfte zu unterstützen. Es gibt schon heute verschiedene Lernprogramme, die KI nutzen. Diese werden jedoch meistens nur von einzelnen Individuen verwendet. KI-Modelle wie GPT haben das Potenzial, die Bildung grundlegend zu revolutionieren. Sie könnten personalisierte Lerninhalte bereitstellen, individuelle Schwächen identifizieren und gezielt darauf eingehen. Zudem können sie administrative Aufgaben übernehmen, sodass Lehrkräfte mehr Zeit für die persönliche Betreuung der Schüler haben. Durch den Einsatz von KI in der Bildung könnte das Lernen effektiver und effizienter gestaltet werden, was letztlich zu besseren Lernergebnissen führen würde.
\section{Sollte unser Schulsystem angepasst werden?}
Das Potenzial der KI bestreitet niemand. Trotzdem mögen wir Menschen drastische Veränderungen nicht. Ein Beispiel ist die Erfindung des Taschenrechners. Anfangs wurde diese Erfindung nicht akzeptiert und ihr Wert wurde heruntergespielt. Viele Menschen protestierten und fanden die Idee nicht gut. Heutzutage können wir uns Schulen ohne Taschenrechner nicht mehr vorstellen. Ich glaube, dass es mit KI ähnlich sein wird.(Fun Fact: Ich wusste nicht, dass Texas Instruments den ersten Taschenrechner entwickelt hat.)
\section{Fazit}
Es gibt viele Vor- und Nachteile der Verwendung von KI in der Bildung. Ich denke, dass das Bildungssystem mit der richtigen Ausführung und passenden Erklärungen umgekrempelt werden könnte. Jeder darf im eigenen Rhythmus lernen. Unterschiedlich begabte Persone könnten ihre Defizite ausgleichen.
Haben Sie sich je vorgestellt, wie personalisierte Lernpläne jedem Schüler die Möglichkeit geben können, alles zu erlernen, was sie brauchen, ohne sich dem Tempo der Klasse anzupassen? Lehrer könnten entlastet werden indem sie Schülerfortschritte überwachen und individuelle Hilfe anbieten dürfen. Des Weiteren kann die Verwaltung auch von künstlicher Intelligenz besser bewältigt werden, wodurch mehr Zeit für direkten Kontakt mit Schülern bleibt.
AI könnte auch einen Beitrag zur inklusiven Bildung leisten. Spezialausbildungsprogramme zum Nutzen solcher Studenten würden entwickelt sein; diese sind genau an deren Bedürfnissen angepasst. In derselben Weise könnten Studieninhalte interaktiver und spannender gemacht werden, was die Motivation der Schüler steigert.
Natürlich gibt es auch Herausforderungen und Bedenken wie Datenschutz oder Ethik bei der Nutzung KI. Diese Faktoren sollen jedoch sorgfältig berücksichtigt werden.
\printbibliography

\end{document}
