\documentclass{report}

\usepackage[ngerman]{babel}
\usepackage[utf8]{inputenc}
\usepackage[T1]{fontenc}
\usepackage{hyperref}
\usepackage{csquotes}
\usepackage[a4paper]{geometry}

\usepackage[
    backend=biber,
    style=apa,
    sortlocale=de_DE,
    natbib=true,
    url=false,
    doi=false,
    sortcites=true,
    sorting=nyt,
    isbn=false,
    hyperref=true,
    backref=false,
    giveninits=false,
    eprint=false]{biblatex}
\addbibresource{../references/bibliography.bib}

\parskip=1em
\parindent=0em
\title{Ethik im Umgang mit Daten}
\author{Levent Bal}
\date{\today}


\begin{document}

\maketitle

\abstract{
In dieser Arbeit werden ethische Aspekte mit umgang der Ki behandelt sowie deren Nutzung in der Schule. Zudem wird ein Überblick darüber gegeben wie Ki funktioniert.

\tableofcontents

\chapter{Einleitung}


    \item \textbf{Wie kann Künstliche Intelligenz (KI) in der Bildung eingesetzt werden?}
    \begin{itemize}
        \item Diese Frage ist zentral, da KI das Potenzial hat den Bildungssektor zu verändern und zu verbessern. Es ist wichtig zu verstehen wie KI zur Unterstützung von Lehrern und Schülern genutzt werden kann.
    \end{itemize}
    
    \item \textbf{Was sind die Vor- und Nachteile von KI?}
    \begin{itemize}
        \item Diese Frage ist entscheidend, um eine ausgewogene Sichtweise auf die Nutzung von KI zu erhalten. Es ist notwendig, sowohl die positiven Aspekte und die negativen Aspekte wie Datenschutz und ethische Bedenken zu betrachten.
    \end{itemize}
    
    \item \textbf{Sollte unser Schulsystem angepasst werden?}
    \begin{itemize}
        \item Diese Frage zielt darauf ab, zu diskutieren, ob und wie das bestehende Schulsystem verändert werden sollte, um die Integration von KI zu ermöglichen und optimal zu nutzen. 
    \end{itemize}
    
    \item \textbf{Weitere Fragen}
    \begin{itemize}
        \item Weitere Fragen, wie die Funktionsweise von KI oder die ethischen Aspekte der Nutzung von KI-Informationen, werden ebenfalls in dieser Arbeit behandelt.
    \end{itemize}


\input{chap_methode.tex}

\section{Etwas mit Quellen}

Etwas mit Änderung hier am Ende.

Wenn ich eine Quelle zitieren möchte, kann ich das ganze einfach am Ende des Satzes machen . Oder wie \citet{example} sagt, auch mitten im Text.

\printbibliography

\end{document}
